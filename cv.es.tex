%%%%%%%%%%%%%%%%%%%%%%%%%%%%%%%%%%%%%%%%%
% Compact Academic CV
% LaTeX Template
% Version 2.0 (6/7/2019)
%
% This template originates from:
% https://www.LaTeXTemplates.com
%
% Authors:
% Dario Taraborelli (http://nitens.org/taraborelli/home)
% Vel (vel@LaTeXTemplates.com)
%
% Modified by:
% José María Martín Luque (https://jmml.me)
%
% License:
% CC BY-NC-SA 3.0 (http://creativecommons.org/licenses/by-nc-sa/3.0/)
%
%%%%%%%%%%%%%%%%%%%%%%%%%%%%%%%%%%%%%%%%%

%----------------------------------------------------------------------------------------
%	PACKAGES AND OTHER DOCUMENT CONFIGURATIONS
%----------------------------------------------------------------------------------------

\documentclass[11pt]{article} % Default document font size
\usepackage[spanish]{babel}
\usepackage[utf8]{inputenc}

%%%%%%%%%%%%%%%%%%%%%%%%%%%%%%%%%%%%%%%%%
% Compact Academic CV
% Structural Definitions
% Version 1.0 (6/7/2019)
%
% This template originates from:
% https://www.LaTeXTemplates.com
%
% Authors:
% Dario Taraborelli (http://nitens.org/taraborelli/home)
% Vel (vel@LaTeXTemplates.com)
%
% License:
% CC BY-NC-SA 3.0 (http://creativecommons.org/licenses/by-nc-sa/3.0/)
%
%%%%%%%%%%%%%%%%%%%%%%%%%%%%%%%%%%%%%%%%%

%----------------------------------------------------------------------------------------
%	REQUIRED PACKAGES AND MISC CONFIGURATIONS
%----------------------------------------------------------------------------------------

\usepackage{graphicx} % Required for including images

\setlength{\parindent}{0pt} % Stop paragraph indentation

%----------------------------------------------------------------------------------------
%	MARGINS
%----------------------------------------------------------------------------------------

\usepackage{geometry} % Required for adjusting page dimensions and margins

\geometry{
	paper=a4paper, % Paper size, change to letterpaper for US letter size
	top=3.25cm, % Top margin
	bottom=4cm, % Bottom margin
	left=3.5cm, % Left margin
	right=3.5cm, % Right margin
	headheight=0.75cm, % Header height
	footskip=1cm, % Space from the bottom margin to the baseline of the footer
	headsep=0.75cm, % Space from the top margin to the baseline of the header
	%showframe, % Uncomment to show how the type block is set on the page
}

%----------------------------------------------------------------------------------------
%	FONTS
%----------------------------------------------------------------------------------------

% \usepackage[utf8]{inputenc} % Required for inputting international characters
% \usepackage[T1]{fontenc} % Output font encoding for international characters

% \usepackage[semibold]{ebgaramond} % Use the EB Garamond font with a reduced bold weight
\usepackage{fontspec}
\setmainfont{PT Sans}

%----------------------------------------------------------------------------------------
%	SECTION STYLING
%----------------------------------------------------------------------------------------

\usepackage{sectsty} % Allows changing the font options for sections in a document

\sectionfont{\fontsize{13.5pt}{18pt}\selectfont} % Set font options for sections
\subsectionfont{\mdseries\scshape\normalsize} % Set font options for subsections
\subsubsectionfont{\mdseries\upshape\bfseries\normalsize} % Set font options for subsubsections

%----------------------------------------------------------------------------------------
%	MARGIN YEARS
%----------------------------------------------------------------------------------------

\usepackage{marginnote} % Required to output text in the margin

\newcommand{\years}[1]{\marginnote{\scriptsize #1}} % New command for adding years to the margin
\renewcommand*{\raggedleftmarginnote}{} % Left-align the years in the margin
\setlength{\marginparsep}{-10pt} % Move the margin content closer to the text
\reversemarginpar % Margin text to be output into the left margin instead of the default right margin

%----------------------------------------------------------------------------------------
%	COLOURS
%----------------------------------------------------------------------------------------

\usepackage[usenames, dvipsnames]{xcolor} % Required for specifying colours by name

%----------------------------------------------------------------------------------------
%	LINKS
%----------------------------------------------------------------------------------------

\usepackage[bookmarks, colorlinks, breaklinks]{hyperref} % Required for links

% Set link colours
\hypersetup{
	linkcolor=blue,
	citecolor=blue,
	filecolor=black,
	urlcolor=MidnightBlue
}
 % Include the file specifying the document structure and styling

% Set PDF meta-information
\hypersetup{
	pdftitle={José María Martín Luque - Curriculum vitae},
	pdfauthor={José María Martín Luque}
}

\usepackage{fancyhdr}
\fancyhf{}
\renewcommand{\headrulewidth}{0pt}
\cfoot{
	\scriptsize
	Última actualización: \today
}

%----------------------------------------------------------------------------------------

\begin{document}

\pagestyle{fancy}

%----------------------------------------------------------------------------------------
%	CONTACT AND GENERAL INFORMATION
%----------------------------------------------------------------------------------------

{\LARGE\bfseries José María Martín Luque}
\bigskip\bigskip\medskip

\ifdefined\emailphone
  \input{email-phone.es.tex}
\fi
Página web: \href{https://jmml.me}{jmml.me}\\
Github: \href{https://github.com/jmml97}{@jmml97}\\
LinkedIn: \href{https://linkedin.com/in/jmml97}{@jmml97}\\

%----------------------------------------------------------------------------------------
%	EDUCATION
%----------------------------------------------------------------------------------------

\section*{Educación}

\years{2015-2020} Grado en Ingeniería Informática, Universidad de Granada\\
	{\color{gray}\itshape Enfocado en el desarrollo web y la administración de sistemas}\\

\years{2015-2020} Grado en Matemáticas, Universidad de Granada\\


%----------------------------------------------------------------------------------------
%	PROJECTS ACTIVITIES
%----------------------------------------------------------------------------------------

\section*{Proyectos y otras actividades}

\years{2019-2020} Trabajo de fin de grado\\
{\color{gray}El objetivo del trabajo era estudiar e implementar en SageMath el algoritmo de Peterson-Gorenstein-Zierler para códigos cíclicos sesgados. El texto y el código están disponibles en \href{https://github.com/jmml97/tfg}{GitHub}.}\\
{\color{gray}\itshape — Calificado con matrícula de honor}\\

\years{2018-2019} Director de Comunicación de la Delegación General de Estudiantes de la Universidad de Granada\\
{\color{gray} Responsable del diseño y la ejecución de campañas de comunicación dirigidas a 47~000 estudiantes. A cargo también de las redes sociales y las relaciones con la prensa local.}\\

Colaborador en \href{https://github.com/libreim/apuntesdgiim}{apuntesDGIIM}\\ 
{\color{gray} Proyecto estudiantil dedicado a la creación de libros de texto gratuitos y de código abierto para asignaturas de Informática y Matemáticas.

Además de colaborar en la elaboración de los textos, creé las plantillas utilizadas, además de la página web del proyecto, así como de su compilación y distribución.}\\

Desarrollo web\\
{\color{gray} He diseñado y desarrollado: \href{https://cielos.es}{cielos.es}, \href{https://herminialuque.com}{herminialuque.com} y mi \href{https://jmml.me}{página web personal}.}\\

%----------------------------------------------------------------------------------------
%	LANGUAGES & SKILLS
%----------------------------------------------------------------------------------------
\vspace{1cm}

\begin{minipage}[t]{0.5\textwidth}
	\section*{Idiomas}

	\begin{tabular}{@{}ll}
	\textsc{Castellano:} & Lengua materna\\

	\textsc{Inglés:} & Fluido (B2)\\

	\textsc{Francés:} & Básico\\
	\end{tabular}\\
\end{minipage}
\begin{minipage}[t]{0.5\textwidth}
	\section*{Habilidades}

	\begin{tabular}{@{}ll}
	\textsc{Programación:} & C++, Python, Swift\\

	\textsc{Desarrollo web:} & HTML, CSS, PHP, Javascript\\

	\textsc{Otras:} & Docker, MySQL, \LaTeX\\

	\end{tabular}\\
\end{minipage}

\end{document}
